\section{Intégration}
À plus forte raison que les dérivées, il est parfois compliqué de calculer
des intégrales analytiques. Dans ce TP nous écrirons des programmes
calculant numériquement des approximations d'intégrales.

La méthode utilisée est basée sur le théorème fondamental de l'analyse.
Celui-ci faisant le lien entre primitive et intégrale au sens de Riemann :
\begin{equation}
\begin{split}
\int_a^b f(x)dx &= F(b)-F(a) \\
&= \lim_{n\rightarrow\infty} \sum_{i=0}^{n-1}\frac{(b-a)}{n} f(a+i\frac{(b-a)}{n})\\
&= \lim_{n\rightarrow\infty} \sum_{i=0}^{n-1} h_n f(x_i)
\end{split}
\end{equation}
Où $dF = f(x)dx$, $h_n=\frac{(b-a)}{n}$, et $x_i=a+ i\frac{(b-a)}{n} = a+ih_n$.

\subsection{Méthode des rectangles}
\begin{enumerate}
\item À partir de la formule ci-avant, 
écrire une fonction capable de calculer $\int_a^b f(x)dx$
pour toute fonction $f(x)$ définie sur l'intervalle $[a\,;\,b]$. 
La fonction d'intégration aura pour arguments $f$, $a$, $b$, et $n$ \cf{function.py}.
\item Écrire la même fonction mais prenant cette fois pour argument la 
valeur de $h$. La fonction commencera par adapter au plus juste 
cette valeur à l'intervalle $[a\,;\,b]$ reçu.
\item Écrire une fonction utilisant la précédente
 pour calculer $N$ intégrales de $a$ à $x$, où les valeurs de $x$
seront uniformément répartie entre $a$ et $b$ \cf{mesh.py et loop.py}. 
Cette fonction
aura donc pour arguments $f$, $a$, $b$, $N$, et $h$ ; elle retournera
deux listes \cf{returnList.py}. 
Celle des valeurs de $x$ et celle de l'intégrale correspondante
$\int_a^xf(t)dt$.


Tester 
cette fonction 
en prenant $f=\cos$, $a=0$, $b=\pi$, $N=100$, et $h=0.01$. 
\item Tracer la fonction ainsi obtenue en même temps que $\sin$ \cf{plot.py}.
\item Calculer l'écart moyen entre cette fonction et $\sin(x)$ \cf{loop.py}.
\item Tracer l'évolution de cet écart en fonction de $h$ sur un 
graphique loglog. Aux mêmes points tracer $f(x)=x$, puis comparer. Utiliser des 
valeurs de $h$ appartenant à une suite géomètrique de raison 4 et de valeur initiale $10^{-4}$. Faire le tracé pour des valeurs de $h$ allant jusqu'à $10^{-2}$.
\end{enumerate}

\subsection{Intervalle de confiance}
Pour une fonction strictement croissante sur un intervalle, le calcul précédent
est une minoration de l'intégrale. Inversement pour une fonction strictement décroissante. À partir de cette constatation, il est possible d'écrire un programme donnant un encadrement de l'intégrale d'une fonction. 
\begin{enumerate}
\item  Écrire une fonction calculant 
 $$\sum_{i=1}^n h_n f(x_i),$$
puis comparer les intégrations réalisés avec cette fonction à celles réalisées avec 
$$\sum_{i=0}^{n-1}h_n f(x_i).$$
Ces méthodes sont appelés méthodes des rectangles à droite et à gauche respectivement.
\item Vérifier sur quelques exemples que pour des portions de fonctions monotones, l'une des intégrations donne une majoration de l'intégrale alors que l'autre donne une minoration. 
\item Écrire une fonction calculant simultanément 
une majoration et une minoration approximative de 
l'intégrale \cf{if.py}. L'algorithme sera le suivant : 
Pour chaque sous-intervalle de largeur $h$, 
la valeur de la fonction à droite sera comparée à celle à gauche. 
La plus grande valeur du produit $hf$ servira à la majoration, 
la plus petite à la minoration.
\end{enumerate}

\subsection{Méthodes d'ordre plus élevé}
Bien entendu, l'intégrale d'une fonction peut être approximée par des méthodes plus efficaces. 
\begin{enumerate}
\item La première consiste à calculer non pas l'air des rectangles 
sous/sur la courbe,
mais celle des trapèzes. 
\begin{equation}
\begin{split}
\int_a^b f(x)dx &\simeq \sum_{i=0}^{n-1} h_n \frac{(f(x_i)+f(x_{i+1})}{2}\\
                &\simeq \frac{h_n}{2} \left(f(a)+f(b)
+ 2\sum_{i=1}^{n-1} f(x_i) \right).
\end{split}
\end{equation}
Programmer cette méthode et comparer sur un graphique loglog 
sa précision à nombre d'évaluation égales
de la fonction intégrée à celle des méthodes des rectangles. 

Pour cela calculer l'écart entre l'intégrale exacte et son calcul approximatif en fonction des valeurs de $h$ pour une seule intégrale. Par exemple $\int_{\sqrt{2}}^{\sqrt{7}} \exp(x) dx$.
\item La méthode d'ordre directement supérieur est appelée méthode de Simpson composite.
Elle consiste à calculer l'aire sous chaque intervalle comme l'aire de la parabole passant par les deux points aux extrémités et le point du milieu : 
\begin{equation}
\int_a^b f(x)dx \simeq  \frac{h}{6}\sum_{i=0}^{n-1} f(x_i) + 4 f(x_{i+1/2}) + f(x_{i+1}) .
\end{equation}

Programmer cette méthode et comparer sa précision en fonction du nombre d'évaluation de la fonction à celle des autres. 
\item En utilisant la même méthode de manière à minimiser les évaluations de la fonction à intégrer on obtient la formule :
 \begin{equation}
\begin{split}
\int_a^b f(x)dx \simeq \frac{h}{48} &\biggl[17f(a) + 59f(a+h)+ 43f(a+2h)+ 49f(a+3h)\biggr.\\
&+48\sum_{i=4}^{n-4}f(x_i)\\
&\biggl.+49f(b-3h)+43f(b-2h)+59f(b-h)+17f(b)\biggr]
\end{split}
 \end{equation}
Ajouter cette formulation à votre comparatif ; en faisant attention à ce qu'elle ne 
peut être utilisée pour $n<8$.
\item Les quadratures de Gauss-Legendre 
consistent à approximer la fonction à intégrer 
par un polynôme de Legendre. Il en existe de divers degrés $m$, où $m$ 
est aussi le nombre d'évaluation de la fonction sur l'intervalle d'intégration.
 Elles sont
exactes pour l'intégration de polynômes de degrés $2m+1$. Pour réussir un tel 
exploit, il faut choisir des points bien particuliers où évaluer la fonction :
\begin{center}
\begin{tabular}{ccc}
Ordre $m$& Points $p_j$ &Pondération $w_j$\\
\hline\hline
\strut$1$ & $0$ & $2$ \\\hline
\strut$2$ & $\pm \sqrt{1/3}$ & 1\\\hline
\multirow{2}{*}{$3$} &  $0$              & $8/9$\\
                     & $\pm\sqrt{3/5}$   & $5/9$\\\hline
\multirow{2}{*}{$4$} &  $\pm\sqrt{\frac{3}{7}-\frac{2}{7}\sqrt{\frac{6}{5}}}$              & $\frac{18+\sqrt{30}}{36}$\\
                     &  $\pm\sqrt{\frac{3}{7}+\frac{2}{7}\sqrt{\frac{6}{5}}}$              & $\frac{18-\sqrt{30}}{36}$\\\hline
\multirow{3}{*}{$5$} & $0$ & $\frac{128}{225}$\\
                     &  $\pm\frac{1}{3}\sqrt{5-2\sqrt{\frac{10}{7}}}$ & $\frac{322+13\sqrt{70}}{900}$\\
                     &  $\pm\frac{1}{3}\sqrt{5+2\sqrt{\frac{10}{7}}}$ & $\frac{322-13\sqrt{70}}{900}$\\

\end{tabular}
\end{center}
Ces approximations sont plutôt déstinées à être employées pour calculer une intégrale sans subdiviser l'intervalle d'intégration. Nonobstant en opérant la subdivision 
on obtient la formule : 
\begin{equation}
\int_a^b f(x)dx \simeq \sum_{i=0}^{n-1} \sum_{j=0}^{m-1} w_j f(x_i+\frac{h}{2}(1+p_j))
\end{equation}
Comparer ces quadratures aux autres intégrateurs déjà programmés \cf{listElement.py  et listOfList.py}.

 



\end{enumerate}
