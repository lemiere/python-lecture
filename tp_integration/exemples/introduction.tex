\section{Dérivation}


\subsection{Mathématiques}
L'objectif du TP est de programmer des fonctions capables de calculer des intégrales de fonction ; d'étudier les paramètres en influençant la précision ; et de comparer cette dernière propriété entre plusieurs méthodes. 

\begin{enumerate}
\item Présenter les intégrales au sens de Riemann : 
$$\int_a^b f(x) dx = \lim_{n\rightarrow\infty} \sum_{i=1}^n \frac{b-a}{n}
f\left( a+i\frac{b-a}{n}\right) $$
\item Présenter les méthodes d'approximation par calcul de l'air d'un rectangle, d'un trapeze, d'une parabole. 

Pour une parabole d'équation $P(x)=ax^2+bx+c$ l'air algébrique sous la parabole entre $x$ et $x+h$ vaut $$\int_x^{x+h} (ax^2+bx+c)dx = ch+\frac{b}{2}(h^2+2xh)+\frac{a}{3}(3x^2h+3h^2x+h^3).$$
Pour trouver les coefficients $a$, $b$, et $c$ il faut résoudre le système d'équations : 
\begin{align}
f(x) &= a x^2 + b x + c \\
f(x+h/2) &= a (x+h/2)^2 + b (x+h/2) + c \\
f(x+h) &= a (x+h)^2 + b (x+h) + c \\
\end{align}
En les réinjectant dans l'équation de $\int_x^{x+h} P(x) dx$ on retrouve la formule de Simpson.
Toutefois il est plus simple de se contenter de vérifier que $\frac{h}{6}\left[f(x) + 4f(x+h/2) + f(x+h)\right] =\frac{h}{6}\left[P(x) + 4P(x+h/2) + P(x+h)\right] = \int_x^{x+h} P(x) dx.$ 
\item Présenter les théorèmes fondamentaux de l'analyse (Certaines fonctions sont la dérivée de leur intégrale. Certaines fonction sont l'intégrale de leur dérivée). Soit :
$$ \int_a^b f(x) dx = F(b) -F(a).$$
\item Montrer comment le DL à l'ordre 1 de F(x+h) donne une formule $O(h^2)$ de rectangles.
\item Montrer comment une méthode de point milieu et un DL à l'ordre 1 ($\int_{x-h}^{x+h}f(x)dx$) donne une formule $O(h^3).$  
\end{enumerate}

\subsection{Programmation}
L'une des méthodes de calcul (Quadrature de Gauss) utilise un tableau à deux dimensions. Des exemples de manipulations de telles tableaux sont donnés dans le script {\tt listOfList.py}.

