%%\section{Pair ou impair}

\section{Exercices}

\subsection{Résolution d'équation du second degré}

Cet exercice consiste à résoudre une équation du second degré de la forme
$a.x^2+b.x+c = 0$.
Les variables \texttt{a}, \texttt{b} et \texttt{c} sont les coefficients de l'équation
avec \texttt{a} non-nul. \texttt{x} est l'inconnue à calculer.

Pour rappel, la résolution de cette équation passe par le calcul du discriminant
$\Delta = b^2-4.a.c$.\\
Par la suite, si le résultat du discriminant est :
\begin{description}
  %%
\item[positif] il existe deux solutions :
\begin{eqnarray*}
  x_1 = \frac{-b-\sqrt{\Delta}}{2.a} \qquad \textrm{et} \qquad x_2=\frac{-b+\sqrt{\Delta}}{2.a}
\end{eqnarray*}
%%
\item[nul] il existe une solution unique :
\begin{eqnarray*}
  x_1 = \frac{-b}{2.a}
\end{eqnarray*}
%%
\item[négatif] il existe des solutions dans l'ensemble complexe :
\begin{eqnarray*}
  x_1 = \frac{-b-i\sqrt{\Delta}}{2.a} \qquad \textrm{et} \qquad x_2=\frac{-b+i\sqrt{\Delta}}{2.a}
\end{eqnarray*}
%%
\end{description}

Vous suivrez ces quelques étapes pour parvenir aux solutions de \texttt{x}


\begin{enumerate}

\item Affectez différentes variables que vous utiliserez une valeur par défaut.
%%
\item Faites intervenir un utilisateur auquel vous demanderez  d'entrer des valeurs aux
  variables \texttt{a}, \texttt{b} et \texttt{c}
%%
\item Affichez chacune des variables pour vérification.
%%
\item Une fois les variables affectées d'une valeur, calculez le discriminant.\\
  Pensez à importer les modules nécessaires.
%%
\item Testez le résultat du discrimant puis en fonction de cette valeur, informez,
  par l'affichage d'un message à l'écran, le nombre de solutions existantes.
%%
\item Calculez les solutions de \texttt{x} et affichez le résultat.
  %%
\item Vérifiez par calcul les résultats fournis par votre programme.

\end{enumerate}
