\section{Une (trop) courte introduction \`a Python}

\subsection{Qu'est ce q'un langage de programmation ?}

Un langage informatique est un langage destin\'e \`a d\'ecrire l'ensemble des actions cons\'ecutives qu'un ordinateur doit ex\'ecuter.
Ce langage se doit d'\^etre compr\'ehensible par l'homme.
Toutefois, le langage utilis\'e par l'ordinateur est appel\'e langage machine.
Il s'agit d'une suite de 0 et de 1 (\textit{binaire}).
Il est donc n\'ecessaire de transformer le langage de programmation en langage
machine exploitable par le processeur.

Les langages de programmation peuvent \^etre s\'epar\'es en deux grandes cat\'egories : compil\'es ou interpr\'et\'es.
\begin{description}
\item[Langage interpr\'et\'e] Un programme \'ecrit dans un langage interpr\'et\'e a besoin d'un programme auxiliaire,
  \textit{l'interpr\'eteur} pour traduire au fur et \`a mesure les instructions du programme.
  La cons\'equence est qu'un langage interpr\'et\'e est en moins rapide qu'un langage compil\'e.


\item[Langage compil\'e] Un programme  dit 'compil\'e' va \^etre traduit en fichier \textit{binaire} une fois pour toutes
  par un autre programme, \textit{le compilateur}.
  Le fichier g\'en\'er\'e est \textit{\'ex\'ecutable}, c'est \`a dire qu'il n'aura plus besoin d'un programme autre
  que lui pour s'ex\'ecuter (par exemple, le langage C++).
  La cons\'equence est que le programme doit est compil\'e \`a chaque modification du code.
\end{description}

Python est un langage interpr\'et\'e, c'est-\`a-dire que chaque ligne de code
est lue puis interpr\'et\'ee afin d'\^etre ex\'ecut\'ee par l'ordinateur.


\subsection{\`A savoir sur Python}


Python est un langage de programmation, dont la première version,
créé par Guido van Rossum, est sortie en 1991.
La Python Software Foundation est une organisation à but
non lucratif particulièrement dévouée créée en 2001.
Ce langage a \'et\'e baptis\'e ainsi en hommage \`a la troupe de
comiques les \textit{Monty Python}.

Ce langage, facile \`a apprendre, dispose d'une large communaut\'e d'utilisateur.
Vous trouverez de nombreux exemples mis \`a disposition en ligne ainsi que de multiples
fonctionnalit\'es pr\^etes \`a l'usage.

Lors de la cr\'eation de la Python Software Foundation, en 2001, le langage Python est pass\'e par une suite de versions
que l'on a englob\'ees dans l'appellation Python 2.x (2.3, 2.5, 2.6).
Depuis le 13 f\'evrier 2009, la version 3.0.1 est disponible.
Cette version casse la compatibilit\'e ascendante qui pr\'evalait lors des derni\`eres versions.

Pour cela, durant ce cours, nous pr\'eciserons, autant que possible, la compatibilit\'e des fonctionalit\'es entre les version 2.x et 3.0.

\subsection{Pourquoi Python ? }

Pour tout un tas de raisons que vous découvrirez un langage simple et clair :
\begin{itemize}
\item[$\bullet$] code facile à lire et intuitif,
\item[$\bullet$] visuellement épuré,
\item[$\bullet$] syntaxe minimaliste facile à apprendre,
\item[$\bullet$] moins de lignes de code, moins de bugs, plus facile à entretenir.
\end{itemize}

De plus, techniquement :
\begin{itemize}
\item[$\bullet$] pas besoin de définir le type de variables, les arguments de
  fonction ou les types de retour.
\item[$\bullet$] pas besoin d’allouer et de désallouer
explicitement la mémoire pour les variables et les tableaux de données. Pas de
bugs de fuite de mémoire.
\item[$\bullet$] Pas besoin de compiler le code. L’interpréteur Python lit et
exécute directement le code python.
\end{itemize}

Le principal avantage est la facilité de programmation, minimisant le temps
nécessaire pour développer, déboguer et maintenir le code.
De plus la programmation modulaire et orientée objet favorise la réutilisation du code.
Une vaste bibliothèque standard et une grande collection de modules
complémentaires, scientifiques et autres sont partagées.

Toutefois, le langage de programmation interprété et typé dynamiquement (comme Python)
est plus lent par rapport aux langages de programmation compilés (comme le C/C++).


\subsection{Installation}

La plupart du temps, Python est install\'e par d\'efaut sur le syst\`eme d'exploitation que vous utiliserez.
Toutefois, vous trouverez l'ensemble des indications sur le site officiel de Python\footnote{https://www.python.org/}.

\begin{description}
\item [Linux/Ubuntu] https://www.python.org/downloads/source\\
  sudo apt-get install python3

\item [MacOS] https://www.python.org/downloads/mac-osx/\\

\item[Windows] https://www.python.org/downloads/windows\\
  http://www.formation-django.fr/python/comment-installer-python.html
\end{description}



\subsection{Quelques liens à connaitres}

\begin{itemize}
\item[] https://openclassrooms.com/courses/apprenez-a-programmer-en-python
\item[] https://github.com/jakevdp/PythonDataScienceHandbook
\end{itemize}

\subsection{Vos premi\`eres lignes de Python}


Pour commencer, il vous faudra d\'emarrer l'interpr\'eteur de commande Python dans un terminal.

\vskip 2pt
\begin{center}
  \begin{myterminalbox}[colback=gray!10]{Terminal}
\begin{verbatim}
  dupont@scssvr:~\$ python
  Python 2.7.12 (default, Nov 19 2016, 06:48:10)
  GCC 5.4.0 20160609 on linux2
  Type "help", "copyright", "credits" or "license" for more information.
  >>>
\end{verbatim}

  \end{myterminalbox}
\end{center}

Le triple chevron $>>>$ est l'invite de commande de Python, cela signifie que l'interpr\'eteur attend une commande.

\vskip 2pt
\begin{center}
  \begin{myterminalbox}[colback=gray!10]{Terminal}
\begin{verbatim}
>>> print("Hello World!")
Hello World!
\end{verbatim}
  \end{myterminalbox}
\end{center}



Pour sortir, faire \fbox{\tt ~Ctrl~+~d}.


Ouvrez de nouveau l'interpr\'eteur et, cette fois, tentez votre premier calcul :
\vskip 2pt
\begin{center}
    \begin{myterminalbox}[colback=gray!10]{Terminal}
\begin{verbatim}
  >>> 6*7
  42
  >>>
\end{verbatim}

    \end{myterminalbox}
\end{center}


Ainsi utilis\'e, l'interpr\'eteur est un programme interactif dans lequel vous entrer des commandes qui seront ex\'ecut\'ees
\`a chaque validation lorsque vous tapez sur \fbox{\tt ~Entr\'ee}.

Bien s\^ur, l'interpr\'eteur pr\'esente vite des limites lorsque vous souhaitez effectuer une s\'erie d'instruction plus complexe.
Vous pourrez alors enregistrer ces instructions dans un fichier portant l'extension \textit{.py}.
Chaque ligne correspond \`a une instruction. L'ordinateur va \'ex\'ecuter ces instructions ligne par ligne.
Vous trouverez un exemple ci-dessous : (\`A noter le \#! est appel\'e le sha-bang)

\vskip 2pt
\begin{center}
  \begin{myterminalbox}[colback=gray!10]{test\_0.py}
\begin{verbatim}
  #!/usr/bin/python

  # Lemiere Yves
  # 22/08/2016
  # Ceci est un commentaire inutile

  print("Hello World !")  # ceci affiche une chaine de characteres
  print(6*7)              # ceci affiche le resultat de l'operation
\end{verbatim}

  \end{myterminalbox}
\end{center}

\subsection{Les mots clefs}

Vous devez garder à l'esprit que certains mots sont absolument réservés avec python.

\begin{center}
\begin{tabular}{ccccc}
and    & assert  & break & class & continue\\
def    & del 	 & elif  & else  & except\\
exec   & finally & for 	 & from  & global\\
if     & import  & in    & is 	 & lambda\\
not    & or 	 & pass  & print & raise\\
return & try 	 & while & yield &       \\
\end{tabular}
\end{center}

\vfill
