\section{Exemples}

\subsection{Découverte de l'interpréteur python}

Pour ces quelques questions préliminaires, utilisez l'interpréteur de commande Python.

\begin{enumerate}

\item Dans un terminal, tapez la commande \texttt{python} et commentez ce que vous voyez à l'écran.

\item Utilisez l'invite de commande de cette console python comme une calculatrice puis commentez.

\item Affichez votre premier texte à l'écran en utilisant la fonction print : \\
   \texttt{print("Hello World")}

\item Quittez l'interpréteur python en effectuant  \texttt{Ctrl+D}

\end{enumerate}


\subsection{Un peu plus qu'une calculatrice}

Pour cette série de commande, vous commencerez par ouvrir l'interpréteur de commande python.

\begin{enumerate}

\item Déclarez plusieurs variables nommées \texttt{a},\texttt{b},\texttt{c},\texttt{d}, \texttt{e} et \texttt{f}
            puis initialisez les avec les valeurs suivantes : 5, 13, 17, 3.14159, 'bob' et 'bonjour'

\item Affichez à l'écran la valeur de chacune des variables en utilisant la commande \texttt{print}.

\item Affichez le type de chaque variable avec la commande \texttt{type()}.\\
  Exemple : \texttt{type(a)}

\item Faites la somme des variables \texttt{a} et \texttt{b} et placez le
  résultat dans une variable \texttt{sum} puis affichez le résultat.

\item Affectez la valeur 666 à la variable \texttt{a} puis affichez de nouveau la valeur de la variable \texttt{sum}.

\item Faites le calcul de \texttt{a} divisé par \texttt{b} puis commentez le résultat.

\item Faites le calcul de \texttt{a} divisé par \texttt{d} puis commentez le résultat.

\item Faites la différence de \texttt{f} et de \texttt{c} puis commentez le résultat.

\item Que donne la somme des variables \texttt{f} et \texttt{e} ?

\item Proposez une méthode pour échanger le contenu des variables \texttt{a} et \texttt{b} (sans utilisez de langage de programmation).
  Ensuite réalisez cette méthode en utilisant les commandes python que vous connaissez.

\end{enumerate}


\subsection{Le premier programme Python}

Afin d'exécuter une suite d'instructions plus complexes,
comme tout langage informatique, on peut enregistrer ces
instructions dans un fichier (programme).

Gardez à l'esprit que tout ligne commençant par \# est une ligne de commentaire qui ne
sera donc pas executée

\begin{enumerate}
\item Éxecuter votre programme en passant dans le terminal la commande \texttt{python script\_0.py} puis commentez.

\item Ouvrez avec \texttt{emacs} le programme \texttt{script\_0.py}.

\item La première ligne de votre script (version très simplifiée d'un programme)
  est ce que l'on nomme le shebang (ou shabang).

\item Que se passe t'il après avoir supprimé la première ligne du programme ?

\item Que permet de faire l'instruction  \texttt{$\backslash$n} ?

\item Décryptez la ligne \#12 du programme \texttt{script\_0.py}.

\item Détaillez les commandes en charges de l'affichage des variables \texttt{a}, \texttt{b} et \texttt{c}.


\end{enumerate}


\subsection{Un peu d'interaction avec votre programme}


Assez rapidement, vous chercherez à proposer vos programmes à des utilisateurs.
Nous allons donc introduite la notion de fonction principale \texttt{main()} que vous trouverez dans l'exemple
\texttt{conditions.py}.
De plus, il sera nécessaire que l'utilisateur puisse interagir avec une interface sans avoir à intervenir
dans les algorithmes que vous aurez mis en place.


\begin{enumerate}
\item Éxecuter votre programme en passant dans le terminal la commande \texttt{python conditions.py} puis commentez.

\item Repérez le début du programme principale et son utilisation par \texttt{main()}.

\item Détaillez la fonctionnalité de la fonction \texttt{input()}

\item De quel type sont les variables \texttt{name} et \texttt{number} ?

\item Votre première condition et donc premier bloc d'instruction se trouve entre les lignes 32 et 37.\\
  Expliquez la différence entre \texttt{if}, \texttt{elif} et \texttt{else}.

\item Entre les lignes 40 et 47, découvrez les mots cléfs \texttt{and}, \texttt{or} et \texttt{not}.\\
  Proposez une solution pour améliorer ce bloc d'instruction et ainsi prendre en
  compte le dernier cas, ignoré jusqu'à maintenant.


\end{enumerate}

\subsection{Un peu de mathématique}

\begin{enumerate}
\item Calculez la racine carrée de la valeur absolue de -1024.
  Comment feriez vous en utilisant python ?

\item Il \textit{suffit} d'importer le module \texttt{numpy} et d'utiliser ensuite
  toutes les fonctions définies dans ce module.
  Pour simplifier, un module est un bout de code que l'on trouve dans un fichier
  mettant à disposition un ensemble de fonctionnalité autour d'un thème commun.
  Pour connaitre l'ensemble des possibilités du module \texttt{math}, il faut taper la commande
  \texttt{help("numpy")} dans l'interpréteur de commande python.


\item Ouvrez le programme \texttt{scientifique.py} et lisez les commandes successives.
  La ligne 6 permet d'importer l'ensemble des fonctionnalités du module \texttt{numpy}

\item Ce qui fait la force du langage python est, entre autre, le nombre important de module.
  Découvrez les modules \texttt{time} ou \texttt{random}.


\end{enumerate}
