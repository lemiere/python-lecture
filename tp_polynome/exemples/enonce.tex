%%\section{Exemples}

\subsection{Découverte de l'interpréteur python}

Pour ces quelques questions  préliminaires, utilisez l'interpréteur de
commande Python.

\begin{enumerate}

\item Dans un terminal, tapez la commande \texttt{python} et commentez
  ce que vous voyez à l'écran.

\item Utilisez l'invite de commande  de cette console python comme une
  calculatrice puis commentez.

\item Affichez votre premier texte  à l'écran en utilisant la fonction
  print : \\ \texttt{print("Hello World")}

\item Quittez l'interpréteur python en effectuant \texttt{Ctrl+D}

\end{enumerate}


\subsection{Un peu plus qu'une calculatrice}

Pour   cette  série   de   commandes,  vous   commencerez  par   ouvrir
l'interpréteur de commandes python.

\begin{enumerate}


\item        Déclarez        plusieurs        variables        nommées
  \texttt{a},\texttt{b},\texttt{c},\texttt{d},      \texttt{e}      et
  \texttt{f} puis initialisez les avec  les valeurs suivantes : 5, 13,
  17, 3,14159, 'bob' et 'bonjour'.\\ Exemple : \texttt{a = 5}



\item  Affichez  à l'écran  la  valeur  de  chacune des  variables  en
  utilisant la commande \texttt{print}.

\item  Affichez   le  type  de   chaque  variable  avec   la  commande
  \texttt{type()}.\\ Exemple : \texttt{type(a)}

\item Faites la somme des variables \texttt{a} et \texttt{b} et affectez
  le  résultat  à  une   variable  \texttt{sum}  que vous afficherez.

\item Affectez la valeur 666 à la variable \texttt{a} puis affichez de
  nouveau la valeur de la variable \texttt{sum}.

\item  Faites  le calcul  de  \texttt{a}  divisé par  \texttt{b}  puis
  commentez le résultat.

\item  Faites  le calcul  de  \texttt{a}  divisé par  \texttt{d}  puis
  commentez le résultat.

\item  Faites  la  différence  de \texttt{f}  et  de  \texttt{c}  puis
  commentez le résultat.

\item Que donne la somme des variables \texttt{f} et \texttt{e} ?

\item  Proposez une  méthode pour  échanger le  contenu des  variables
  \texttt{a}   et   \texttt{b}   (n'utilisez   pas   de   langage   de
  programmation).   Ensuite réalisez  cette méthode  en utilisant  les
  commandes python que vous connaissez.

\end{enumerate}


\subsection{Le premier programme Python}

Afin d'exécuter  une suite  d'instructions plus complexe,  comme dans
de nombreux langages  informatiques, on  peut enregistrer  ces
instructions  dans un fichier (on parle de programme \emph{source}
ou \emph{script} ou encore \emph{macro}).


Gardez à l'esprit  que toute séquence de caractères sur une ligne
commençant par le caractère \#  est interprétée comme un commentaire
et ne sera donc pas exécutée.

\begin{enumerate}
\item Dans le terminal, choisir le répértoire de travail approprié.
  Exécutez  le programme \texttt{script\_0.py} en  passant dans le
  terminal  la commande  \texttt{python script\_0.py}  ou la  commande
  \texttt{./script\_0.py}, comparez et commentez.

\item Ouvrez avec \texttt{emacs} le programme \texttt{script\_0.py}.

\item La première ligne de  votre script (version très simplifiée d'un
  programme) est, ce que l'on nomme, le \emph{shebang}.

\item Que  se passe-t-il après  avoir supprimé  la première  ligne du
  programme ?

\item Analysez la ligne \#12 du programme \texttt{script\_0.py}.

\item Que se passe t'il lorsque le code \texttt{$\backslash$n} est imprimé ?

\item Détaillez les commandes en  charge de l'affichage des variables
  \texttt{a}, \texttt{b} et \texttt{c}.


\end{enumerate}
\begin{figure}  
  \lstinputlisting{../exemples/script_0.py}
  \caption{Contenu du fichier \texttt{script\_0.py}}
  \label{polynome_script_0}
\end{figure}





\subsection{Un peu d'interaction avec votre programme}




Nous  allons introduire  la  notion de  fonction
principale   \texttt{main()}  que   vous   trouverez  dans   l'exemple
\texttt{conditions.py}.


\begin{enumerate}


\item  Ouvrez avec  \texttt{emacs} le  programme \texttt{conditions.py}
  puis repérez le début du programme principal et son utilisation par
  \texttt{main()}.

\item Exécuter votre programme en passant dans le terminal la commande
  \texttt{python conditions.py} puis commentez.


\item Détaillez la fonctionnalité de la fonction \texttt{input()}.

\item De quels types sont les variables \texttt{name}, \texttt{input\_number},
et \texttt{number} ? Il est possible de demander à l'interpréteur Python cette information grâce à la fonction \texttt{type()}.


\item Le premier test conditionnel amorce le bloc d'instructions
  situé entre  les lignes 28  et 33.\\ Expliquez la  différence entre
  \texttt{if}, \texttt{elif} et \texttt{else}.

%2017-09-07 FM: !!! pas de or no not !!!
\item  Entre   les  lignes  36   et  43,  découvrez  les   mots-clefs
  \texttt{and}, \texttt{or}  et \texttt{not}.\\ Proposez  une solution
  pour ajouter  un bloc  d'instructions et ainsi  prendre en  compte le
  dernier cas, ignoré jusqu'à maintenant.


\end{enumerate}


\begin{figure}  
  \lstinputlisting{../exemples/conditions.py}
  \caption{Contenu du fichier \texttt{conditions.py}}
  \label{polynome_conditions}
\end{figure}




\subsection{Un peu de mathématiques}

\begin{enumerate}
\item Calculez la racine carrée de la valeur absolue de -1024.
  Comment feriez-vous en utilisant python ?


\item Il \textit{suffit} d'importer le module \texttt{numpy} et d'utiliser ensuite
  des fonctions définies dans ce module.
  Pour simplifier, un module est un bout de code que l'on trouve dans un fichier
  mettant à disposition un ensemble de fonctionnalités autour d'un thème commun.
  Pour connaître l'ensemble des possibilités du module \texttt{numpy}, il faut taper la commande
  \texttt{help("numpy")} dans l'interpréteur de commandes python.


\item Ouvrez le programme \texttt{scientifique.py} et lisez les commandes successives.
  La ligne 6 permet d'importer l'ensemble des fonctionnalités du module \texttt{numpy}.


\item Ce qui fait la force du langage python est, entre autres, le nombre important de modules disponibles.
  Découvrez les modules \texttt{time} ou \texttt{random} à l'aide de la fonction \texttt{help()}.



\end{enumerate}


\begin{figure}  
  \lstinputlisting{../exemples/scientifique.py}
  \caption{Contenu du fichier \texttt{scientifique.py}}
  \label{polynome_scientifique}
\end{figure}


\subsection{Manipuler les chaînes de caractères}


\begin{enumerate}

\item Ouvrez avec  \texttt{emacs} le  programme \texttt{chaine.py} puis
  suite à l'analyse du programme, définissez le type de chaque variable.

\item Expliquez ce que fait la ligne 23 de votre programme.

\item Comparez le nombre d'éléments constituant les variables \texttt{list\_of\_word} et  \texttt{sentence}.

\item Détaillez les fonctionalités des méthodes \texttt{split()}, \texttt{replace()} et \texttt{count()}.

\item Proposez une solution pour accéder au cinquième élément de la variable \texttt{list\_of\_word}.

\item Comparez les deux méthodes d'affichage ligne~33 et ligne~37 puis faites un choix.


\end{enumerate}


\begin{figure}  
  \lstinputlisting{../exemples/chaine.py}
  \caption{Contenu du fichier \texttt{chaine.py}}
  \label{polynome_chaine}
\end{figure}