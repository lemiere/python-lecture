\section{Évaluation de fonctions}
L'évaluation de fonctions transcendantes
comme ($\sin $, $\tan^{-1}$, $\exp $\ldots) et même de
certaines autres fonctions ($\sqrt{\strut}$, $x^y$\ldots) 
en fonction de leurs arguments n'est
pas immédiat à la main. Les ordinateurs et autres 
calculatrices rendent ces calculs accessibles. Dans ce
TP nous nous proposons d'explorer l'évaluation de ces 
fonctions à l'aide de leur développement en série de Taylor.



\subsection{Tangente inverse}

Le développement en série de Taylor de la fonction tangente inverse autour de 0 
s'écrit : 
\begin{equation}
\arctan(x) = \sum_{i=0}^\infty (-1)^i\frac{x^{2i+1}}{2i+1}
=x -\frac{1}{3}x^3 + \frac{1}{5}x^5-\frac{1}{7}x^7+\ldots
\end{equation}

\begin{enumerate}
\item Une condition évidente pour que cette série converge est $|x|<1$.
 Programmer 4 fonctions calculant ce développement limité aux ordres 3, 5, 7, et 9
d'$\arctan(x)$. 
\item Tracer ces 4 fonctions, et $\arctan(x)$ 
sur l'intervalle $[-1,1\,;\,1,1]$ \cf{polynomesEtSeries.py}.
Ensuite tracer la valeur absolue de la différence 
de chacune de ces fonctions avec $\arctan(x)$. 
Pour quelles valeur de $x$ semble-t-il facile 
d'évaluer numériquement la fonction $\arctan(x)$ ? 
\item Écrire une fonction capable de calculer $\arctan(x)$ à partir de 
son développement limité à l'ordre $2n+1$ ; où $n$ 
sera un paramètre de la fonction.
Cette fonction suffit-elle à calculer $\arctan(x)$ ? 
\item Pour faciliter le calcul on peut utiliser la formule : 
\begin{equation}\label{arctanred2}
\arctan(x) = 2 \arctan\left(\frac{x}{1+\sqrt{1+x^2}}\right).
\end{equation}
Pour quelles valeurs de $x$, cette formule peut-elle faciliter le calcul ? 
\item Programmer une fonction faisant usage 
deux fois successivement de la formule~\ref{arctanred2} 
avant de calculer $\arctan(x)$ en utilisant 
un développement d'ordre $2n+1$. 
\item Évaluer le bon fonctionnement de cette fonction sur 
l'intervalle $[-7,\,;\,7]$. Notre fonction pourra-t-elle effectivement 
caculer $\arctan(x) \,\, \forall x \in [-\infty\,;\,\infty]$ ?
\item Il est facile d'améliorer encore les résultats précédents 
grâce à la formule : 
\begin{equation}
\arctan(x) = \frac{|x|\pi}{x2}-\arctan(1/x) 
\end{equation}
Programmer une nouvelle fonction utilisant les développements précédents et
cette nouvelle transformation quand $|x|>1$ \cf{if.py}. Pour 
quelles valeurs de $x$ l'erreur est-elle maximale ? 
\item Déterminer l'ordre de développement à utiliser pour que 
l'erreur maximale soit inférieure à $10^{-13}$.
\end{enumerate}

\section{Cosinus}
Le développement en série de Taylor de la fonction cosinus autour de $0$ s'écrit : 
\begin{equation}
\sin(x) = \sum_{i=0}^\infty \frac{(-x^{2})^i}{2i!} 
\end{equation}
\begin{enumerate}
\item Tracer ce développement aux ordres 2, 4, et 6 sur l'intervalle $[-2\pi\,;\,2pi]$.
\item Améliorer le calcul en tenant compte du fait que la fonction cosinus est
périodique de période $2\pi$. 
\item Écrire une fonction calculant le DL à un ordre arbitraire, et arrêtant automatiquement le calcul lorsque la valeur absolue du rapport 
du dernier terme calculé sur le premier terme de la série est inférieure à $10^{-16}$. Tester toujours sur le même intervalle. Quelle est la valeur maximale de $i$ utiliseée ? 
\item En utilisant le fait que $\cos(x) = -\cos(\pi-x)$, améliorer le calcul. Quelle
est désormais la valeur maximale prise par $i$ ?
 \end{enumerate}
\section{Exponentielle}
\begin{enumerate}
\item Établir le développement limité autour de 0 de la fonction $\exp(x)$.
\item Tracer l'erreur commise en utilisant ce DL à l'ordre 10 pour estimer $\exp(x)$
pour des valeurs de $x \in [-4\,;\,4]$.
\item Expliquer la dissymétrie observée entre 
valeurs de $x$ positives et négatives. 
\item Améliorer la fonction en utilisant le fait que $\exp(x) = 1/\exp(-x)$.
\item Faciliter encore le calcul en utilisant la formule $\exp^n(x/n) = \exp(x)$. 
Pour ce faire, on écrira une fonction puissance 
récursive \cf{recursion.py} capable
de calculer la puissance $n^\textrm{ième}$ d'un nombre, et la valeur de $n$ 
sera choisie de sorte que $x/n$ soit inférieure à $1$. 
\end{enumerate}

