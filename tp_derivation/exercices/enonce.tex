\section{Derivation}
Il arrive que la dérivation analytique d'une fonction ne soit pas facilement
calculabe. Ce TP survol les méthodes de dérivation numérique.
\subsection{Précision}
Cet exercice vise à étudier les erreurs commises lors de l'emploi
d'un méthode de dérivation numérique.
\begin{enumerate}
\item Écrire un fonction capable de calculer la dérivée 
d'une fonction (de $\mathbb{R}  \longrightarrow  \mathbb{R}$)  en 
un point $x$ quelconque en utilisant la formule : 
$$f'(x)= \frac{f(x+h)-f(x-h)}{2h}\,;$$
où $h$ est un paramètre qui pourra être choisi lors de l'appel de cette fonction
\cf{function.py}.
\item Utiliser cette fonction pour calculer $\frac{d\cos(x)}{dx}$ 
sur mille points répartis homogènements sur $[0\,;\,\,2\pi]$
\cf{mesh.py}.
Ensuite tracer la fonction $\cos(x)$ et les points calculés sur
un graphique
\cf{plot.py}.
\item Sur les mille points, calculer la moyenne de la valeur
 absolue de la différence
\cf{loop.py}.
\item Sur une échelle log/log, tracer l'évolution de cette différence 
en fonction de $h$. Les valeurs choisies pour $h$ suivront une suite 
géométrique de raison $2^\frac{1}{2}$ et de valeur initiale $10^{-6}$ \cf{loop.py}. 
On utilisera les 25 premières valeurs.
Qu'observe-t-on ? 
Comment choisir la meilleur valeur de $h$ pour dériver numériquement ? 
\item Tracer la dérivée numérique et la dérivée 
analytique sur un même graphique. Peut-on observer facilement la différence ? 
\item Tracer la différence ente dérivée numérique et calcul exact. 
\item Comparer la forme de la fonction obtenue avec 
$-\alpha \sin(x)$ en choisissant de manière appropriée $\alpha$.
Expliquer ce qui est observé. 
\end{enumerate}

\subsection{Algorithmes de dérivation}
De nombreux algorithmes de dérivation numérique existent. Cette section vise à
en programmer quelques-uns et à les comparer.
\begin{enumerate}
\item Programmer les fonctions de dérivation utilisant les formules : 
$$f'(x)= \frac{-f(x+2h)+8f(x+h)-8f(x-h)+f(x-2h)}{12h}\,;$$ 
$$f'(x)= \frac{f(x+3h)-9f(x+2h)+45f(x+h)-45f(x-h)+9f(x-2h)-f(x-3h)}{60h}.$$
Les coefficients de ces fonctions sont calculés de manière à annuler de plus en
plus de termes du développement limité de la fonction $f$.
\item Des formules non-symétriques existent également. En voici quelques-unes :
 $$f'(x) = \frac{f(x+h)-f(x)}{h}\,;$$
$$f'(x)=\frac{-3f(x)+4f(x+h) -f(x+2h)}{2h}\,;$$
$$f'(x)=\frac{-15f(x)+16f(x+h) -f(x+4h)}{12h}\,;$$
 Programmer les fonctions de dérivation correspondantes. Ces fonctions peuvent être utilisées avec $h>0$ ou $h<0$.
\item À nombre d'évaluation de la fonction $f$ égale, comparer la précision moyenne de
chacune de ces méthodes sur un graphique (pour des valeurs de $h$ comprises entre $0,001$ et $1$).
\end{enumerate}



\subsection{Ordres supérieurs}
Il est possible d'obtenir des dérivées d'ordre supérieur par la même méthode (seconde, troisième\ldots).
\begin{enumerate}
\item Programmer une fonction calculant la dérivée à l'ordre 2 d'une fonction en 
utilisant l'algorithme $f'(x)= \frac{f(x+h)-f(x-h)}{2h}\,;$ pour obtenir 
des dérivées première, puis calculant la dérivée seconde à partir de deux points ainsi obtenus. 
\item Programmer une fonction calculant la dérivée seconde à l'aide de la formule :
$$f'(x)= \frac{f(x+h)+f(x-h)-2f(x)}{h^2}.$$
Comparer l'évolution de l'erreur de ces deux méthodes en fonction de $h$. Puis réaliser
une comparaison \og{} sur papier \fg{} des deux méthodes. 
\end{enumerate}

\subsection{Dérivées de série de points régulière}
Ces méthodes de dérivation numérique peuvent être utiles pour dériver des fonctions 
dont la forme analytique est complexe et pour lesquelles on dispose juste 
d'une série de points. À titre d'exercice nous allons générer une telle série 
de points dans un fichier, puis calculer la dérivée.

\begin{enumerate}
\item Écrire un programme générant un fichier contenant deux colonnes \cf{file.py}. Dans la première
on trouvera les valeurs de $x$ et dans la seconde celles de $f(x)$ ; où $f$ sera une fonction de votre choix. Générer 100 points sur un intervalle $[a\,;\,b]$ de votre choix. 
\item Écrire un programme capable de relire ce fichier.
\item Combien de valeurs de la dérivée peut-on calculer avec une formule de dérivation centrée ? 
\item Écrire une fonction capable de calculer la dérivée de la fonction échantillonnée dans le fichier pour chacun des points.
\end{enumerate}

\subsection{Dérivées de séries de points irrégulière}
Reproduire le même travail, mais cette fois avec une série de points distribuées de manière irrégulière sur l'intervalle $[a\,;\,b]$. Il faudra donc calculer à chaque fois de nouveaux coefficients pour la dérivation. 
