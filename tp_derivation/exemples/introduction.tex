\section{Dérivation}


\subsection{Mathématiques}
L'objectif du TP est d'abord de programmer des formules de dérivations numériques. 
Elles sont toutes basées sur le développement limité des fonctions. 

\begin{enumerate}
\item Dans un premier temps montrer que 
$$\frac{f(x+h)-f(x-h)}{2h} = f'(x) + O(h^3).$$

\item La meilleure valeur de $h$ pour calculer la dérivée analytique est conditionnées d'une part par la différence entre  $f(x+h)$ et $f(x-h)$ qui doit être aussi grande que possible et $h^3$ qui doit être aussi petit que possible. 

\item Dans le cas de la fonction choisie $f'''(x) = -\sin x$. L'étude du développement limité nous montre que le $O(h^3) = f'''(x) h^3/3! + O(h^5)$. C'est ce qu'on retrouve en faisant la différence entre la dérivée numérique et la dérivée réelle. 

\item En choisissant un échantillonnage régulier de fonction $f$ autour de $x$, on peut écrire les développement limités de $f$ en ces points et trouver les coefficients tels que la somme de ces DL annule un maximum de termes après $f'(x)$. Cette méthode permet de trouver les formules d'ordre supérieur, ou non symétrique.

\item L'évolution de la précision moyenne de ces formules de dérivation est fonction du premier terme non annullé dans la somme des DL. 


\item De la même manière on peut obtenir des formules d'ordre supérieur en annullant dans la sommation les coef. des dérivées d'ordre inférieur à celle souhaitée. Par ailleurs, plus on annulera de coefficients de dérivées d'ordre supérieur plus la formule sera précise.

\item Si l'on désire appliquer ces méthodes à des séries irrégulières il faut, soit prendre une formule dyssymétrique à deux points, soit recalculer à chaque fois les coefficiens annulant des dérivées.  
\end{enumerate}
\subsection{Programmation}
Rien de nouveau. 

