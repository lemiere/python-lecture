\section{Tirage pseudo-alétoire ou comment faire du hazard avec une machine déterministe}


\subsection{Rappel à propos des modules}


Les modules sont des collections de fonctions, variables et autres constantes qui
peuvent être réutilisées pour écrire des programmes sans devoir les redéfinir soi-
même. Le nouveau programme a simplement besoin de déclarer qu’il importe
les fonctions, variables etc. d’un module, en spécifiant le nom de celui-ci, avant
de pouvoir les utiliser.

\subsection{Introduction aléatoire}

\subsubsection{Qu'est ce que le pseudo-aléatoire}


Un générateur de nombres pseudo-aléatoires est un algorithme qui génère une suite de nombres
présentant certaines propriétés du hasard.
Les nombres sont supposés être aussi indépendants que possible les uns des autres, et il est donc
difficile de repérer des groupes de nombres qui suivent une certaine règle.


\subsubsection{Pourquoi a t on besoin de nombres aléatoires ?}

Pour simuler un modèle stochastique, il est nécessaire de disposer d’une source
de nombres \textit{au hasard} susceptibles de jouer en pratique le rôle
des variables aléatoires qui interviennent dans la définition du modèle.

Voici une liste absolument pas exhaustive des domaines nécessitant une tirage aléatoire.

\begin{itemize}
\item[$\bullet$] Jeux de hasard
\item[$\bullet$] Génération de mot de passe
\item[$\bullet$] Cryptographie
\item[$\bullet$] Simulation
\end{itemize}


\subsubsection{Existe t il différent tirage pseudo-aléatoire ? }


La plupart du temps, le tirage pseudo-aléatoire consiste à produire un nombre
avec une probabilité uniforme entre 0 et 1.





\vskip 2pt
\begin{center}
  \begin{myterminalbox}[colback=gray!10]{random.py}
\begin{verbatim}


\end{verbatim}
  \end{myterminalbox}
\end{center}
