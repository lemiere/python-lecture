\subsection{Même chance pour tous}

Le programme \texttt{random\_0.py}  contient deux fonctions distinctes
permettant de tirer des nombres aléatoires entiers ou réels.

\begin{enumerate}
\item Modifiez le programme afin  de sauvegardez dans un fichier texte
  l'ensemble des valeurs générées.

\item Après  plusieurs exécutions  du programme, comparez  les données
  produites  et mettez  en \oe  uvre  une solution  pour améliorer  le
  caractère aléatoire du programme.
  
\item Tracez  un histogramme de  distribution des valeurs  entières et
  des valeurs réelles produites.

\item  Calculez  la  valeur  moyenne de  la  distribution  de  valeurs
  entières  générées puis  comparez  à la  valeur  moyenne attendue  à
  partir des paramètres entrés par l'utilisateur.

\item Notez  $\varepsilon$ l'écart  entre les deux  moyennes calculées
  précédemment puis tracez l'évolution de $\varepsilon$ en fonction du
  nombre de valeurs générées.

\end{enumerate}
%%

\subsection{Description du tirage aléatoire}

Sur  la base  du programme  \texttt{random\_and\_stat.py}, vous  allez
étudier les caractéristiques d'un tirage aléatoire uniforme.


\begin{enumerate}
\item  Calculez la  valeur moyenne  des valeurs  entières générées  et
  comparez à la valeur attendue.

\item  Calculez  $\varepsilon_1$  la  différence  entre  les  moyennes
  calculées à la question précédente.

\item Écrivez une fonction pour accueillir l'algorithme que vous venez
  d'écrire.

  
\item Tracez $\varepsilon_1$  en fonction du nombre  de valeurs tirées
  aléatoirement $N$ avec $N \in [10:10^5]$.
  
\item Calculez la quantité de nombres tirés par valeur (nombre de $a$,
  $a+1$, \dots,  $b-1$, $b$)  et comparez à  la valeur  attendue. Pour
  chaque  valeur  calculez l'écart  entre  valeur  attendue et  valeur
  obtenue ($\varepsilon_2$).

\item Tracer l'histogramme de  distribution de l'écart $\varepsilon_2$
  pour chaque valeur.

\end{enumerate}


\subsection{L'aléatoire à la base de la simulation Monte-Carlo}

Bien que  la simulation en  physique n'ait rien  à voir avec  les jeux
pratiqués   dans   la   ville   éponyme,  vous   allez   utilisés   le
\textit{hasard}   pour  introduire   une  approche   statistique  d'un
problème.  Nous verrons dans cet exemple une méthode simple pour faire
un choix équiprobable dans une liste de particules élémentaires.  Pour
cela vous utiliserez l'exemple \texttt{random\_a\_list.py}.

\begin{enumerate}
\item Proposez à l'utilisateur de mélanger la liste de particules puis
  utilisez la méthode adéquat en cas de réponse positive.  %%
\item Affichez  la liste produite  après mélange puis vérifiez  que la
  liste a bien été mélangée.  %%
\item  Proposez  à  l'utilisateur de  sélectionner  aléatoirement  une
  particule dans la liste. Répétez cette étape $N$ fois.  %%
\item   Vérifiez  que   pour   $N$  \textit{grand},   le  tirage   est
  équiprobable.  %%
\item Proposez  une méthode pour tirer  aléatoirement deux particules
  distinctes.  %%
\end{enumerate}


\subsection{Tout est question de distribution}

Jusqu'à  maintenant, vous  avez étudiez  des distributions  aléatoires
uniformes, cela signifie que chaque valeur tirée a la même probabilité
d'exister.  Utilisez les  exemples \texttt{random\_distribution.py} et
\texttt{random\_2D.py} pour écrire votre  propre programme et répondre
aux questions suivantes.


\begin{enumerate}
\item Comparez  et commentez  les différentes  distributions (uniform,
  triangular, gauss).  Calculez les  moyennes, écart type et extremum.
  %%
\item Évaluez les  temps d'exécution des différentes  méthodes pour le
  même nombre de tirage.  %%
\item Écrivez votre propre programme pour :
  \begin{itemize}
  \item[$\ast$] Tirer  aléatoirement selon deux distributions  ($x$ et
    $y$) uniformes $N_{tot} \times N_{tot}$ valeurs comprises entre -1
    et 1.
  \item[$\ast$] Pour un couple $(x,y)$, calculer la valeur $r$ tel que
    $r^2 = x^2+y^2$
  \item[$\ast$] Stocker dans  uniquement les valeurs de $x$  et de $y$
    pour lesquels $r \inf 1$ et  affichez le nombre de valeurs gardées
    ($N_{sel}$).
  \item[$\ast$] Calculer  le rapport de $N_{sel}/N_{tot}$  et comparer
    cette valeur à $\frac{\Pi}{4}$
  \item[$\ast$] Évaluer l'écart entre ces  deux valeurs en fonction du
    nombre de tirage effectués.
  \end{itemize}
  %%
\item Sans  nécessairement écrire l'algorithme, proposez  une solution
  alternative pour tirer une distribution  de points dans un cercle de
  rayon 1.


  
\end{enumerate}


\vfill
