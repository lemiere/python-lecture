\subsection{Le mouvement brownien}

Le  mouvement brownien  correspond  à la  description mathématique  du
mouvement  aléatoire d'une  particule immergée  dans un  fluide et soumise
uniquement      aux     chocs      avec      les     molécules      du
fluide\footnote{https://fr.wikipedia.org/wiki/Mouvement\_brownien}.
Cette exercice consiste à générer (simuler) un tel mouvement et à en
analyser certaines propriétés.


Pour  faire simple,  vous déplacerez  aléatoirement un  point
sur un pavage carré à deux dimensions supposé infini.


\begin{enumerate}
  %%
\item Créez les fonctions suivantes :
  \begin{description}
  \item[choose\_direction]  Pour choisir  aléatoirement une  direction
    dans la liste suivante : Ouest, Sud, Est, Nord.
  \item[prepare\_move]     Pour      produire     les     valeurs
    (\texttt{delta\_x},\texttt{delta\_y})  à partir  d'une direction
    (\texttt{Ouest}, \texttt{Sud},  \texttt{Est},
    \texttt{Nord})\\ Ex : Pour une direction \texttt{Nord} $\rightarrow$ \texttt{delta\_x} = 0 et
    \texttt{delta\_y} = +1.
  \item[move] Pour renvoyer  les nouvelles coordonnées
    (\texttt{new\_x},\texttt{new\_y})
    à partir  d'un mouvement (\texttt{delta\_x},\texttt{delta\_y})
     à  appliquer aux
    coordonnées initiales (\texttt{x},\texttt{y}).
  \item[simulate]  Pour effectuer  $N$ fois  les opérations  précédentes ---
    c'est-à-dire $N$ \emph{itérations}  consistant à choisir  une
    direction, preparer le mouvement dans  un espace à deux dimensions,
    puis  effectuer  le  mouvement ---  puis  renvoyer  les  listes  des
    positions (\texttt{x},\texttt{y}) calculées lors des $N$ mouvements.
  \item[distance] Pour calculer la  distance entre deux points
    (\texttt{x1},\texttt{y1})
    et (\texttt{x2},\texttt{y2}).
  \end{description}
  %%
\item Dans le programme  principal, effectuez les opérations suivantes
  :
  \begin{itemize}
  \item[$\ast$] Générez les coordonnées \texttt{x\_init} et \texttt{y\_init}
    d'une particule au point (0,0).
  \item[$\ast$]  Effectuez une simulation complète  de $N$ itérations. Au terme
    de cette  simulation, calculez la  distance $d$ entre  la position
    initiale et la dernière position.
  \item[$\ast$]   Réalisez $M$  fois  l'étape précédente,  calculez la  moyenne
    $\bar{d}$  des distances  $d$  et tracez  la  distribution de  ces
    distances.
  \item[$\ast$]   Réalisez plusieurs  fois l'étape  précédente et  comparez les
    distributions obtenues et leur évolution en fonction de $N$.
  \end{itemize}


\end{enumerate}


%%\subsection{La marche au hasard}
