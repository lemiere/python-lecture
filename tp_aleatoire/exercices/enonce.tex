\subsection{Le mouvement brownien}

Le  mouvement brownien  correspond  à la  description mathématique  du
mouvement  aléatoire d'une  particule immergé  dans un  fluide soumise
uniquement      aux     chocs      avec      les     molècules      du
fluide\footnote{https://fr.wikipedia.org/wiki/Mouvement\_brownien}.


Pour  faire simple,  vous déplacerez  aléatoirement un  point dans  un
espace discret et infini à deux dimensions.


\begin{enumerate}
  %%
\item Créez les fonctions suivantes :
  \begin{description}
  \item[choose\_direction]  Pour choisir  aléatoirement une  direction
    dans la liste suivante : Ouest, Sud, Est, Nord
  \item[prepare\_mouvement]     Pour      produire     les     valeurs
    (delta\_x,delta\_y)  à partir  d'une direction  (Ouest, Sud,  Est,
    Nord)\\ Ex : Pour une direction Nord $\rightarrow$ delta\_x = 0 et
    delta\_y = +1
  \item[move] Pour renvoyer  les nouvelles coordonnées (new\_x,new\_y)
    à partir  d'un mouvement (delta\_x,delta\_y) à  appliquer au point
    (x,y).
  \item[simulate]  Pour effectuer  $N$ fois  les opérations  suivantes
    (appelé  par  la  suivante   \textit{itérations})  :  choisir  une
    direction, preparer le mouvement dans  un espace à deux dimensions
    puis  effectuer  le  mouvement.   Puis  renvoyer  les  listes  des
    positions (x,y) calculées lors des $N$ mouvements
  \item[distance] Pour calculer la  distance entre deux points (x1,y1)
    et (x2,y2)
  \end{description}
  %%
\item Dans le programme  principal, effectuez les opérations suivantes
  :
  \begin{itemize}
  \item[$\ast$] Générez les coordonnées \texttt{x\_init} et \texttt{y\_init}
    d'une particule au point (0,0).
  \item[$\ast$]  Effectuez une simulation complète  de $N$ itérations. Au terme
    de cette  simulation, calculez la  distance $d$ entre  la position
    initiale et la dernière position.
  \item[$\ast$]   Réalisez $M$  fois  l'étape précédente,  calculez la  moyenne
    $\bar{d}$  des distances  $d$  et tracez  la  distribution de  ces
    distances.
  \item[$\ast$]   Réalisez plusieurs  fois l'étape  précédente et  comparez les
    distributions obtenues.
  \end{itemize}


\end{enumerate}


%%\subsection{La marche au hasard}
